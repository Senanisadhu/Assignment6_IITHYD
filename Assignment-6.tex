\documentclass[a4paper,12pt]{article}
\usepackage{tkz-euclide}
\usepackage{gensymb}
\usepackage[utf8]{inputenc}
\usetikzlibrary{math}
\usepackage{graphicx}
\title{ASSIGNMENT-6}
\author{SENANI SADHU}
\date{\today}
\begin{document}
	\maketitle
	\pagenumbering{roman}
	\newpage
	\section{Draw JUMP with JU = 3.5, UM = 4, MP =
		5, PJ = 4.5 and PU = 6.5
}
	\subsection{Solution:-}
	Given,JU = 3.5, UM = 4, MP =5, PJ = 4.5 and PU = 6.5\\
	Therefore required quadrilateral:
	\begin{center}
		\begin{tikzpicture}
			[scale=2,>=stealth,point/.style={draw,circle,fill = black,inner sep=0.5pt},]
			
			
			%Quadrilateral sides JU, UM, MP, PJ, PU
			\tikzmath{\j = 4 ; \u = 5; \m = 4.5; \p = 3.5; \d = 6.5; }
			%Rotation angles PUM and JUP
			\tikzmath{\t1=51.75338012165502; \t2=34.7719440319486; }
			%
			%Labeling points
			\node (U) at (0, 0)[point,label=below left:$U$] {};
			\node (M) at (\j, 0)[point,label=below right:$M$] {};
			\node (J) at ({\t1+\t2}:\p)[point,label=above right:$J$] {};
			\node (P) at (\t1:\d)[point,label=above right:$P$] {};
			
			%Foot of perpendicular
			
			\draw (J) --  node[left] {$\textrm{p}$}(U) --  node[below] {$\textrm{j}$}(M) --  node[right] {$\textrm{u}$}(P) --  node[above] {$\textrm{m}$}(J);
			\draw (U) -- node[above left] {$\textrm{d}$}(P);
			
			%Drawing and marking angles
			\tkzMarkAngle[fill=orange!50,size=0.5cm,mark=](M,U,P)
			\tkzMarkAngle[fill=green!50,size=0.4cm,mark=](P,U,J)
			\tkzLabelAngle[pos=0.65](M,U,P){$\theta_1$}
			\tkzLabelAngle[pos=0.65](P,U,J){$\theta_2$}
		\end{tikzpicture}
	\end{center}
\subsection{Output of Python code:-}
\begin{figure}[htp]
	\centering
	\includegraphics[width=10cm]{triangle-2.30.png}
	\caption{Fig generated using python}
\end{figure}
\newpage
\section{DRAW rhombus BEND such that BN = 5.6,
	DE = 6.5.}
\subsection{Solution:-}
Given, BN=5.6 DE=6.5
Let BN and ED intersect each other at O.\\
Now, diagonals of rhombus bisect each other at right angles.\\\\
Thus, we have\\
ON=$\frac{BN}{2}$=$\frac{5.6}{2}$=2.8\\\\
OE=$\frac{ED}{2}$=$\frac{6.5}{2}$=3.25\\\\
Since EON is a right angled triangle, by pythagoras theorem, we have\\
$EN^2$=$OE^2$+$ON^2$\\
$EN^2$=10.56+7.84\\
EN=4.29\\\\
Since each side of rhombus are equal\\
Therefore, \\
EN=ND=BD=BE=4.29\\\\
	Therefore required quadrilateral:
\begin{center}
	\begin{tikzpicture}
		[scale=2,>=stealth,point/.style={draw,circle,fill = black,inner sep=0.5pt},]
		
		
		%Quadrilateral sides BE, EN, ND, DB, ED, BN
		\tikzmath{\b = 4.29 ; \e = 4.29; \n = 4.29; \d = 4.29; \j = 6.5; \i=5.6; \s=2.6;}
		%
		%Labeling points
		\node (E) at (0, 0)[point,label=below left:$E$] {};
		\node (N) at (\b, 0)[point,label=below right:$N$] {};
		\node (B) at (-3,3)[point,label=above right:$B$] {};
		\node (D) at (1.1,3)[point,label=above right:$D$] {};
		\node (O) at (0.56,1.54)[point,label=below:$O$] {};
		%Foot of perpendicular
		\draw (B) --  node[left] {$\textrm{d}$}(E) --  node[below] {$\textrm{b}$}(N) --  node[right] {$\textrm{e}$}(D) --  node[above] {$\textrm{n}$}(B);
		\draw (E) -- (D);
		\draw (N) -- (B);
			
		\tkzMarkRightAngle[fill=blue!20,size=.3](B,O,D)
		
	\end{tikzpicture}
\end{center}
\subsection{Output of Python Code:-}
\begin{figure}[htp]
	\centering
	\includegraphics[width=10cm]{triangle-2.35.png}
	\caption{Fig generated using python}
\end{figure}
\end{document}